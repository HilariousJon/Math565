\begin{questions}
\begin{problem}
Find the chromatic polynomial of the $n$-cycle $C_n$ for $n\geq 3$. Find the chromatic polynomial of the $n$-wheel $W_n$ (this is the graph obtained from $C_n$ by adding a
new vertex and joining it to all vertices of $C_n$).
\end{problem}

\begin{solution}
Before we proceed our solution, we first proof two lemmas stated as follows:

\begin{lemma}
Given a graph $G=(V,E)$, and given $e\in E$, if $e$ is not a loop, then its chromatic polynomial satisfy that:
\[
    \chi_G (k) = \chi_{G'_e} (k) - \chi_{G''_e}(k) \\
\]
where $G'_e$ denotes the graph induced by $G$ deleting the edge $e$, and $G''_e$ denotes the graph induced by $G$ contracting the edge $e$.
\label{lem1}
\end{lemma}

\begin{proof}
If $e$ is an edge of $G$, not a loop, we can then break the proper colorings of $G'_e$ into two classes:
\begin{itemize}
    \item Those in which the endpoints of $e$ have different colors. Such case is the proper colorings of $G$, with the chromatic polynomial given as $\chi_G(k)$.
    \item Those in which the endpoints of $e$ have the same colors. Such case is the proper coloring of $G''_e$, with the chromatic polynomial given as $\chi_{G''_e}(k)$.
\end{itemize}

\quad So it follows that:
\[
    \chi_{G'_e} (k) = \chi_{G} (k) + \chi_{G''_e}(k) \\
\]

\quad which is equivalent to:
\[
    \chi_G (k) = \chi_{G'_e} (k) - \chi_{G''_e}(k) \\
\]
\end{proof}

\begin{lemma}
For a tree $T$ on $n$ vertices, its chromatic polynomial is given by:
\[
\chi_T(k) = k(k-1)^{n-1} \\
\]
\label{lem2}
\end{lemma}

\begin{proof}
We shall proceed the proof by induction on the number of vertices:
\begin{enumerate}
    \item \textbf{Base case:} When $n=1$, we see that there are $k$ kinds of different colorings, which aligns with our original statement.
    \item \textbf{Inductive case:} We give our inductive hypothesis as follows: Suppose that when $n=p \geq 1$, we have that $\chi_T(k;p) = k(k-1)^{p-1}$, we want to see that when $n=p+1$, we have $\chi_T(k;p+1) = k(k-1)^{p}$.

    \quad So when $n=p+1$, the tree $T_{p+1} = (V_{p+1}, E_{p+1})$ now have its number of vertices as $p+1$. We choose certain vertex $a\in V_{p+1}$ and delete it from the tree, such that after the deletion, the graph is still a tree. In particular we see that before the deletion, $T_{p+1}$ attains $p+1$ vertices and $p$ edges, and after the deletion, since it is still a tree, it attains $p$ vertices and $p-1$ edges, it means that the deletion only delete a vertex that only connect to one vertex of $T_p$. By \textbf{Inductive Hypothesis}, we see that the chromatic polynomial of $T_p$ is given as $\chi_T(k;p) = k(k-1)^{p-1}$. For vertex $a$, if we insert it back into $T_p$ it only need to have different color with the vertex it connect to, in particular it has $k-1$ number of choices on coloring. So we see that:
    \[
        \chi_{T}(k;p+1) = (k-1)\chi_T(k;p) = (k-1) k(k-1)^{p-1} = k(k-1)^p \\
    \]

    \quad So we see the inductive case also holds.
\end{enumerate}

\quad By above reasoning, we successfully proof the lemma.
\end{proof}

We now \textbf{claim} that:
\begin{equation}
    \chi_{C_n}(k) = (-1)^n(k-1) + (k-1)^n, \text{ } \forall n\geq 3\\
    \label{eq1}
\end{equation}
and
\begin{equation}
    \chi_{W_n}(k) = (-1)^n (k^2 - 2k) + k(k-2)^n, \text{ } \forall n \geq 4\\
    \label{eq2}
\end{equation}

And we shall proceed our proof:
\begin{proof}
We first proof \textbf{Equation} \ref{eq1}. By \textbf{Lemma} \ref{lem1}, we see that:
\[
    \chi_{C_n} (k) = \chi_{C_n'e} (k) - \chi_{C_n''e}(k) \\
\]

\quad where we see that for the case of $C_n$, we have:
\[
\begin{aligned}
    C_n'e &= T_n \\
    C_n''e &= C_{n-1} \\
\end{aligned}
\]

\quad Then we see that:
\[
\begin{aligned}
    \chi_{C_n} (k) &= \chi_{C_n'e} (k) - \chi_{C_n''e}(k) \\    
    \iff \chi_{C_n} (k) & = \chi_{T_n} (k) - \chi_{C_{n-1}}(k) \\
\end{aligned}
\]

\quad By \textbf{Lemma} \ref{lem2}, we have that:
\[
\chi_T(k) = k(k-1)^{n-1} \\
\]

\quad So we see in fact we have:
\[
\begin{aligned}
    \chi_{C_n} (k) & = k(k-1)^{n-1} - \chi_{C_{n-1}}(k) \\
\iff k(k-1)^{n-1} &= \chi_{C_n} (k) + \chi_{C_{n-1}}(k)
\end{aligned}
\]

\quad And it is easy to see that when $n=3$, we have a triangle, and to properly color it with $k$ colors, the chromatic polynomial will be $k(k-1)(k-2)$. In particular, we have:
\[
\chi_{C_3}(k) = k(k-1)(k-2) \\
\]

\quad So by:

\begin{equation}
    \label{eq3}
    \chi_{C_n} (k) + \chi_{C_{n-1}}(k) = k(k-1)^{n-1} \\
\end{equation}

\quad We can then deduce that:
\[
\begin{aligned}
    \chi_{C_n}(k) &= - \chi_{C_{n-1}}(k) + k(k-1)^{n-1} \\
    \chi_{C_{n-1}}(k) &= - \chi_{C_{n-2}}(k) + k(k-1)^{n-2} \\
    \chi_{C_{n-2}}(k) &= - \chi_{C_{n-3}}(k) + k(k=1)^{n-3} \\
    & \vdots \\
    \chi_{C_4} &= - \chi_{C_3}(k) + k(k-1)^3 \\
\end{aligned}
\]

\quad And hence sum up all of the above by coefficient to delete intermediate terms, we can deduce that:
\begin{equation}
\begin{aligned}
    \implies \chi_{C_n}(k) &= k(k-1)^{n-1}-k(k-1)^{n-2} + k(k-1)^{n-3} - k(k-1)^{n-4} + \cdots +\\
    & (-1)^{n-4}k(k-1)^3 + (-1)^{n-5} k(k-1)(k-2)\\
    &= (-1)\sum_{i=3}^{n-1}k(k-1)^i(-1)^{i-n} + (-1)^{n+1}k(k-1)(k-2)\\
    &= (-1)^{n+1}[\sum_{i=3}^{n-1}k(k-1)^i(-1)^i+k(k-1)(k-2)] \\
    &= (-1)^{n+1}k[\sum_{i=3}^{n-1}(1-k)^{i} + (k-1)(k-2)] \\
\end{aligned}
\label{eq4}
\end{equation}

\quad We then denote that:
\[
A = \sum_{i=3}^{n-1}(1-k)^i \\
\]

\quad Then we can deduce that:
\[
\begin{aligned}
    A = (1-k)^3 &+ (1-k)^4 + \cdots + (1-k)^{n-1} \\
\implies    (1-k)A &= (1-k)^4 + \cdots + (1-k)^{n-1} + (1-k)^{n} \\
\implies -kA & = (1-k)^n - (1-k)^3 \\
\implies A &= \frac{(1-k)^3 - (1-k)^n}{k} \\
\end{aligned}
\]

\quad We then plug $A$ back to the \textbf{Equation} \ref{eq4}, and deduce that:
\[
\begin{aligned}
    \chi_{C_n}(k) &= (-1)^{n+1}[-(1-k)^n+(1-k)^3 + k(k-1)(k-2)] \\
    &= (-1)^{n+1}[k(k-1)(k-2) - (k-1)(k-1)^2 - (1-k)^n] \\
    &= (-1)^{n+1}[(k-1)(k^2-2k-(k^2-2k+1)) - (1-k)^n] \\
    &= (-1)^{n+1}[1-k-(1-k)^{n}] \\
    &= (-1)^n(k-1) + (k-1)^n \\
\end{aligned}
\]

\quad And thus we see that:
\[
\chi_{C_n}(k) = (-1)^n(k-1) + (k-1)^n, \text{ } \forall n \geq 3\\
\]

\quad So we successfully proof \textbf{Equation} \ref{eq1}!

\quad Now considering $\chi_{W_n}(k)$, and we have following \textbf{key observations}:
\begin{itemize}
    \item $W_n$ actually consists of $n+1$ vertices, which attains $C_n$ as its subgraph.
    \item The proper coloring of $W_n$ can be proceed by the following procedure:
        \begin{enumerate}
            \item First use a random color from the $k$-color set to color the center of the wheel. Namely first color the vertex that connects to all the rest of the vertices.
            \item then the rest of the vertices form a wheel, and they can be given a proper coloring by $k-1$ different colors.
        \end{enumerate}
\end{itemize}

\quad Thus \textbf{by the key observation}, we can deduce that:
\[
\begin{aligned}
\chi_{W_n}(k) &= k\chi_{C_n}(k-1) \\
\implies \chi_{W_n}(k) &= k((-1)^n (k-2) + (k-2)^n) \\
&= (-1)^n(k^2-2k) + k(k-2)^n \\
\end{aligned}
\]

\quad And thus we see that:
\[
    \chi_{W_n}(k) = (-1)^n (k^2 - 2k) + k(k-2)^n, \text{ } \forall n \geq 4\\
\]

\quad So we successfully proof \textbf{Equation} \ref{eq2}!
\end{proof}
\end{solution}
\end{questions}