\begin{questions}
\begin{problem}
Prove that for all integers $r\geq1$, there is a minimal number $N(r)$ with the following property. If $n\geq N(r)$ and the integers in $\{1,2,\ldots,n\}$ are colored with $r$ colors, then there are three elements $x, y, z$ (not necessarily distinct) with the same color and $x + y = z$. Determine $N(2)$ and show that $N(3) > 13$.
\end{problem}

\begin{solution}
    \begin{proof}
        We take the vertex set as $\{1,2,\ldots,n\}$, and proceed the coloring rule as follow: $\forall \text{ integer }i, j \in [1,n]$, if $i < j$ we color the set $\{i,j\}$ by the color of $(i-j)$. In particular we see in this construction, all vertex pair are colored in one of the $r$ color as required. We claim that the minimal number $N(r)$ we want to find is actually:

        \[
        N(r) =: N(\underbrace{3,3,\ldots,3}_{r \text{ } 3s};2) \\
        \]

        \quad where the right hand side is the Ramsey number. We now look into what happens when $n \geq N(r)$. By \textbf{Ramsey's Theorem}, when $n\geq N(r)$, there exists either a $3$-subset where all of its $2$-subset are colored in the first color, or the second color, or the third color, ..., or the $r$th color. And let's assume that $x < y < z$, with $\{x,y,z\}$ be such existed subset. By assumption, we see $\{x,y\}, \{y,z\}, \{x,z\}$ are colored in the same color. We then further assume:

        \[
        \begin{aligned}
            a &= y-x \\
            b &= z-y \\
            c &= z-x \\
        \end{aligned}
        \]

        \quad By our construction on the rule of coloring the set $\{i,j\}, i,j \in \{1,2,\ldots,n\}$, we see that $a,b,c$ are also colored in the same color. Then observe that:

        \[
        \begin{aligned}
        a + b &= (y-x) + (z-y) = z-x = c \\
        \iff a + b &= c\\
        \end{aligned}
        \]

        \quad thus we see that our construction holds with the requirement, proof done.
    \end{proof}

    \textbf{Determine $\mathbf{N(2)}$?}

    \quad Suppose we color $\{1,2,\ldots,n\}$ by red and blue, so we proceed the coloring step by step. First we color $1$ with red, then by $1+1=2$, we see we have to color $2$ in blue, for $3$, both works, so we color it in blue since if we color it in red, we left no choice for $4$. For $4$ since $2+2=4$, we cannot color it in blue, so we can only color it in red. But then we see that we left no choice of color for $5$, since $1+4=5$ with $1$ and $4$ both colored in red and $2+3=5$ with $2$ and $3$ both colored in blue. Thus we see that $\mathbf{N(2)=5}$. A visualization can be seen as follows, with \textbf{R} denote as colored in red and \textbf{B} denote as colored in blue:

    \[
    \begin{array}{ccccc}
        1 & 2 & 3 & 4 & 5 \\
        \text{R} & \text{B} & \text{B} & \text{R} & \text{\sout{R/B}}
    \end{array}
    \]

    \textbf{Proof that} $\mathbf{N(3) > 13}$.

    \begin{proof}
        We just need to find a counter example for $\{1,2,\ldots,13\}$, which with three color \textbf{R, G, B}, such that any $x<y<z$ with same color, we always have $x+y\neq z$. A counter example can be illustrate as follows, with \textbf{R, G, B} denote as colored in R, G, B respectively.

        \[
        \begin{array}{ccccccccccccc}
             1 & 2 & 3 & 4 & 5 & 6 & 7 & 8 & 9 & 10 & 11 & 12 & 13\\
             \text{R} & \text{G} & \text{G} & \text{R} & \text{B} & \text{B} & \text{B} & \text{B} & \text{B} & \text{R} & \text{G} & \text{G} & \text{R} \\
        \end{array}
        \]

        \quad so our proof is done.
    \end{proof}
    
\end{solution}
\end{questions}