\documentclass[answers]{exam}
\usepackage{amsmath}
\usepackage{amsthm}
\usepackage{hyperref}
\usepackage{amssymb}
\usepackage{amsfonts}
\usepackage{mathrsfs}
\usepackage{stmaryrd}
\usepackage{minted}
\usepackage[margin=0.7in]{geometry}
\usepackage[dvipsnames]{xcolor}
\usepackage{graphicx}
\usepackage{ulem}
\usepackage{thmtools}
\usepackage[most]{tcolorbox}
\renewcommand{\qedsymbol}{$\blacksquare$}
% sans math
% \RequirePackage[cm]{sfmath}
% \RequirePackage{cmbright}
% \RequirePackage{fontspec}
% \setmainfont{CMU Sans Serif}
% \setsansfont{CMU Sans Serif}
% \setmonofont[Scale=.9]{CMU Typewriter Text}

\colorlet{LightGray}{White!90!Periwinkle}
\colorlet{LightOrange}{Orange!15}
\colorlet{LightGreen}{Green!15}

\declaretheoremstyle[name=Theorem,]{thmsty}
\declaretheorem[style=thmsty,numberwithin=section]{theorem}
\tcolorboxenvironment{theorem}{colback=LightGreen}

\declaretheoremstyle[name=Proposition,]{prosty}
\declaretheorem[style=prosty,numberlike=theorem]{proposition}
\tcolorboxenvironment{proposition}{colback=LightOrange}

\declaretheoremstyle[name=Definition,]{defsty}
\declaretheorem[style=defsty,numberlike=theorem]{definition}
\tcolorboxenvironment{definition}{colback=LightOrange}

\declaretheoremstyle[name=Lemma,]{lemsty}
\declaretheorem[style=lemsty,numberlike=theorem]{lemma}
\tcolorboxenvironment{lemma}{colback=LightGreen}

\declaretheoremstyle[name=Problem,]{probsty}
\declaretheorem[style=probsty,numberlike=theorem]{problem}
\tcolorboxenvironment{problem}{colback=LightGray}
